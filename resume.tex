%%% LaTeX Template: Designer's CV
%%%
%%% Source: http://www.howtotex.com/
%%% Feel free to distribute this template, but please keep the referal to HowToTeX.com.
%%% Date: March 2012


%%%%%%%%%%%%%%%%%%%%%%%%%%%%%%%%%%%%%
% Document properties and packages
%%%%%%%%%%%%%%%%%%%%%%%%%%%%%%%%%%%%%
\documentclass[a4paper,12pt,final]{memoir}

% misc
\renewcommand{\familydefault}{bch}	% font
\pagestyle{empty}					% no pagenumbering
\setlength{\parindent}{0pt}			% no paragraph indentation


% required packages (add your own)
\usepackage{flowfram}										% column layout
\usepackage[top=1cm,left=1cm,right=1cm,bottom=1cm]{geometry}% margins
\usepackage{graphicx}										% figures
\usepackage{url}											% URLs
\usepackage[usenames,dvipsnames]{xcolor}					% color
\usepackage{multicol}										% columns env.
\setlength{\multicolsep}{0pt}
\usepackage{paralist}										% compact lists
\usepackage{tikz}

%%%%%%%%%%%%%%%%%%%%%%%%%%%%%%%%%%%%%
% Create column layout
%%%%%%%%%%%%%%%%%%%%%%%%%%%%%%%%%%%%%
% define length commands
\setlength{\vcolumnsep}{\baselineskip}
\setlength{\columnsep}{\vcolumnsep}

% frame setup (flowfram package)
% left frame
\newflowframe{0.2\textwidth}{\textheight}{0pt}{0pt}[left]
\newlength{\LeftMainSep}
\setlength{\LeftMainSep}{0.21\textwidth}
\addtolength{\LeftMainSep}{2\columnsep}

% small static frame for the vertical line
\newstaticframe{1.5pt}{\textheight}{\LeftMainSep}{0pt}

% content of the static frame
\begin{staticcontents}{1}
	\hfill
	\tikz{%
		\draw[loosely dotted,color=RoyalBlue,line width=1.5pt,yshift=0]
		(0,0) -- (0,\textheight);}%
	\hfill\mbox{}
\end{staticcontents}

% right frame
\addtolength{\LeftMainSep}{1.5pt}
\addtolength{\LeftMainSep}{1\columnsep}
\newflowframe{0.7\textwidth}{\textheight}{\LeftMainSep}{0pt}[main01]
\newcolumntype{L}[1]{>{\raggedright\let\newline\\\arraybackslash\hspace{0pt}}m{#1}}
\newcolumntype{C}[1]{>{\centering\let\newline\\\arraybackslash\hspace{0pt}}m{#1}}
\newcolumntype{R}[1]{>{\raggedleft\let\newline\\\arraybackslash\hspace{0pt}}m{#1}}

%%%%%%%%%%%%%%%%%%%%%%%%%%%%%%%%%%%%%
% define macros (for convience)
%%%%%%%%%%%%%%%%%%%%%%%%%%%%%%%%%%%%%
\newcommand{\Sep}{\vspace{1.5em}}
\newcommand{\SmallSep}{\vspace{0.5em}}
\newcommand\tab[1][1cm]{\hspace*{#1}}

\newenvironment{AboutMe}
{\ignorespaces\textbf{\color{RoyalBlue} About me}}
{\Sep\ignorespacesafterend}

\newcommand{\CVSection}[1]
{\Large\textbf{#1}\par
	\SmallSep\normalsize\normalfont}

\newcommand{\CVItem}[1]
{\textbf{\color{RoyalBlue} #1}}


%%%%%%%%%%%%%%%%%%%%%%%%%%%%%%%%%%%%%
% Begin document
%%%%%%%%%%%%%%%%%%%%%%%%%%%%%%%%%%%%%
\begin{document}
	
	% Left frame
	%%%%%%%%%%%%%%%%%%%%
	\begin{figure}
		\includegraphics[width=0.8\columnwidth]{PASS4.jpg}
		\vspace{-7cm}
	\end{figure}
	
	\begin{flushleft}\small
		\textbf{Address:} \\
		2E-604, \\
		DOS Housing Colony,
		Jalahalli, \\
		Bangalore 560013\\
		\textbf{Email-Id:} \\
		rhariharan3096@gmail.com  \\
		\textbf{Contact:}\\
		+91 8792461472
	\end{flushleft}\normalsize
	\framebreak
	
	
	% Right frame
	%%%%%%%%%%%%%%%%%%%%
	\Huge\bfseries {\color{RoyalBlue} R Hariharan} \\
	\small\bfseries B.Tech in Computer Science and Engineering \\
	
	\normalsize\normalfont
	
	
	% Experience
	\CVSection{Objective}
	To understand the laws governing our universe and apply it to intelligent systems, in order to make our lives easier.
	\Sep
	
	% Education
	\CVSection{Education}
	\def\arraystretch{1.5}
	\begin{tabular}{C{2cm} C{3cm} C{2.3cm} C{2cm} C{2.3cm}}
		\hline
		\hline
		\textbf{Degree} & \textbf{Discipline} & \textbf{Institution} &  \textbf{Score} & \textbf{Year of Graduation}\\
		\hline
		\hline
		B.Tech & Computer Science and Engineering (3rd Year) & PES University & 9.28/10 & 2018 \\
		\hline
		Pre-University & 12th Std. & K.L.E Society’s Independent PU college & 92.5\% & 2014\\
		\hline
		All India Secondary School Examination (AISSE) & 10th Std. & Kendriya Vidyalaya IISc & 10/10 & 10/10\\
		\hline
		\hline
	\end{tabular}
	\Sep
	
	\CVSection{Projects}
		\begin{enumerate}
			\item \textbf{HaveASeat - Static website for information on movies} \newline
			{\footnotesize Stylish website that provides information on movies (‘Movie Description’, ‘Review’ and ‘Rating’). All movies are separated based on the language. Used HTML5, CSS, JavaScript and PHP to build the website.}
			\item \textbf{Music Player} \newline
			{\footnotesize JAVA application that provides users with essential features of a music player such as PLAY/PAUSE/STOP, NEXT/PREVIOUS, SEEK, REPEAT and SHUFFLE ALL. Implemented using doubly linked data structure.}
			\item \textbf{HaveASeat - Dynamic website for movie information and booking a ticket} \newline
			{\footnotesize Dynamic website that provides users to view information on movies and can book movies in cinemas. The Database used was a MySQL relational database. The user can filter movies based on language, genre or actors. While booking only those cinemas are shown where the movie is currently running.}
			\item \textbf{Line following and obstacle avoiding robot} \newline
			{\footnotesize The robot follows a black line, if an obstacle is found the robot finds the nearest space/gap to avoid it and then moves forward until it finds a new black line to follow. Robot was built on Arduino platform.}
		\end{enumerate}
	\Sep
	\clearpage
	\framebreak
	\framebreak
	\CVSection{Training \& Internship}
		\begin{itemize}
			\item \textbf{Microsoft Mobile Innovation Lab - Summer Internship 2015} \newline
			{\footnotesize
				\textbf{Location}: PES University, Bangalore \newline
				\textbf{Role}: Intern \newline
				\textbf{Duration}: June - July 2015 \newline
				\textbf{Project Title}: Robot Navigation and Obstacle Avoidance \newline
				\textbf{Abstract}: Developed an algorithm to navigate a robot to its target in an unmapped area amidst obstacles. Implemented the algorithm, on a robotic platform provided by NEX Robotics called FIREBIRD V. The robot was mounted by a Sharp IR sensor to detect obstacles, these sensor values are analysed by the algorithm to find the most efficient path for the robot to reach its target.
			}
			\item \textbf{Microsoft Mobile Innovation Lab - Summer Internship 2016} \newline
			{\footnotesize
				\textbf{Location}: PES University, Bangalore \newline
				\textbf{Role}: Mentor \newline
				\textbf{Duration}: June - July 2016 \newline
				\textbf{Project Title}: Voice Controlled Robot for Travelling Salesman Problem (TSP) Applications \newline
				\textbf{Abstract}: Developed a voice controlled robot on the robotics platform provided by NEX Robotics called FIREBIRD V which can be used as delivery robot. The robot works on an algorithm devised by us to solve the travelling salesman problem. The algorithm requires only half the calculations required by brute force method.
			}
			\item \textbf{Attended IoT BootCamp Organised by Department of Computer Science and Engineering, PES Institute of Technology} 
			
			\item \textbf{Attended Data Analytics Workshop Organised by PES Institute of Technology}
		\end{itemize}
	\Sep
	
	\CVSection{Technical Skills}
		\begin{enumerate}
			\item \CVItem{Programming Languages} \\
			\tab C, JAVA, Python, JavaScript, HTML, PHP, MySQL
			\item \CVItem{Operating Systems} \\
			\tab Windows XP/7/10, UBUNTU 14.04/16.04
			\item \CVItem{Computer Applications} \\
			\tab Microsoft Word, Microsoft Excel, XAMPP, Atmel Studio, Adobe Photoshop, Wireshark, Tex Studio
		\end{enumerate}
	\Sep
	
	\CVSection{Soft Skills}
		\begin{itemize}
			\item \textbf{Student Head} of Microsoft Innovation Lab, PES University
			\item \textbf{Organised} multiple events like:\\
			\tab \textbf{INCITO} - IDEATHON conducted by Microsoft Innovation lab\\
			\tab \tab \textbf{Role}: Member of Logistics and Campaigning Team\\
			\clearpage
			\framebreak
			\framebreak
			\tab \textbf{\#CODE} - 24hr HACKATHON conducted by Microsoft \\\tab \tab Innovation Lab\\
			\tab \tab \textbf{Role}: Member of Logistics, Campaigning and Design\\\tab \tab Team
			\item \textbf{Languages Known}:\\
			\tab Tamil, Kannada, Hindi \& English
			 
		\end{itemize}
	\Sep
	
	\CVSection{Extra-Curricular Activities}
	\Sep
	
	\CVSection{Personal Details}
	\Sep

	% References
	\CVSection{References}
	\Sep
		
	\CVSection{Declaration}
	\Sep
	
	\CVSection{Date}
	\Sep

	
	%%%%%%%%%%%%%%%%%%%%%%%%%%%%%%%%%%%%%
	% End document
	%%%%%%%%%%%%%%%%%%%%%%%%%%%%%%%%%%%%%
\end{document}